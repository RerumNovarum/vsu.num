\documentclass{article}

\usepackage{amsmath}
\usepackage{amsfonts}

\title{Cubic splines}
\author{Sergey Kozlukov}

\begin{document}
\maketitle

\section{Problem}
The problem is that of interpolation with a smooth function
which is piecewise cubic polynomial.
Specifically,
given a table \( \{ (x_i, f_i); i=\overline{0,N} \} \)
of values \( f_i \) of a function at points \( x_i \) of a grid,
the task is to produce a smooth function \( s:\left[x_0,x_N\right]\to\mathbb{R} \)
satisfying the interpolation criteria
\[ s(x_i) = f_i, \]
such that
\( s(x)=s_i(x) \text{ if } x\in\left[x_i, x_{i+1}\right], \)
where \( s_i:\left[x_i, x_{i+1}\right]\to\mathbb{R} \), \( i=\overline{0,N-1} \)
are cubic polynomials:
\[ s_i(x) = a_i + b_i (x-x_i) + \frac12 c_i (x-x_i)^2 + \frac16 d_i (x-x_i)^3, \]
which coincide at shared knots up to second derivative:.
\begin{equation}\label{eq:f_i}
    s_i(x_i) = f_i, s_i(x_{i+1}) = f_{i+1}, \text{ for } i=\overline{0,N-1},
\end{equation}
\begin{equation}\label{eq:df_i}
 s_i'(x_{i+1}) = s_{i+1}'(x_{i+1}), i=\overline{0,N},
\end{equation}
\begin{equation}\label{eq:d2f_i}
 s_i''(x_{i+1}) = s_{i+1}''(x_{i+1}), i=\overline{0,N}.
\end{equation}

That gives us \( 4N-2 \) (independent as turns out later) linear equations for coefficients of polynomials.
To get the last two needed equations
one might require e.g. some boundary constraints.
In this note we'll consider the "clampsed" cubic splines
in which the first derivatives at the boundary are given:
\begin{equation}\label{eq:boundary}
   s_0'(x_0) = f_0',\quad s_{N-1}'(x_N) = f_N'.
\end{equation}

\section{Reduction}
Equations \eqref{eq:f_i}~-~\eqref{eq:boundary} can be rewritten as
\begin{align*}
     & a_i = f_i, \quad i=\overline{0,N-1}, \\
     & h_i b_i + \frac12 h_i^2 c_i + \frac16 h_i^3 d_i = f_{i+1} - f_i, \quad i=\overline{0,N-2}, \\
     & b_i + h_i c_i + \frac12 h_i^2 d_i = b_{i+1}, \quad i=\overline{0,N-2}, \\
     & c_i + h_i d_i = c_{i+1} \\
     & \text{where } h_i = x_{i+1} - x_i, \quad i = \overline{0, N-2}, \\
     & b_0 = f'(x_0), \\
     & b_{N-1} + h_{N-1} c_{N-1} + \frac12 h_{N-1}^2 d_{N-1} = f'(x_N). \\
\end{align*}
Now we can express coefficients \( \{b_i\}, \{d_i\} \) in terms of \( \{c_i\} \)
which in turn can be derived from from a tridiagonal linear system:
\begin{align*}
    & a_i = f_i, \quad i=\overline{0,N-1}, \\
    & d_i = \frac{c_{i+1}-c_i}{h_i}, \quad i=\overline{0,N-2}, \\
    & b_i = \frac{f_{i+1}-f_i}{h_i} - \frac13h_i(c_i + \frac12 c_{i+1}), \quad i=\overline{0,N-2}, \\
    & \frac16 h_i c_i + \frac13 (h_i + h_{i+1}) c_{i+1} + \frac16 h_{i+1} c_{i+2} = \frac{f_{i+2}-f_{i+1}}{h_{i+1}} - \frac{f_{i+1}-f_i}{h_i},
        \quad i=\overline{0,N-3}, \\
    & b_{N-1} = b_{N-2} + h_{N-2} c_{N-2} + \frac12 h_{N-2}^2 d_{N-2}, \\
    & d_{N-1} = \frac{6}{h_{N-1}^3}(f_N - f_{N-1} - h_{N-1} b_{N-1} - \frac12 h_{N-1}^2 c_{N-1}), \\
    & \text{from boundary conditions:} \\
    & a_0 = f_0, \quad b_0 = f_0', \\
    & \frac13 h_0 c_0 + \frac16 h_0 c_1 = \frac{f_1 - f_0}{h_0} - f_0', \\
    & \frac13 h_{N-2} c_{N-2} + (\frac23 h_{N-2}  + \frac12 h_{N-1}) c_{N-1} =
        3\frac{f_N - f_{N-1}}{h_{N-1}} - 2 \frac{f_{N-1} - f_{N-2}}{h_{N-2}} - f_N' , \\
\end{align*}

\end{document}
